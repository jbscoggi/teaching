
\documentclass[warsaw, 9pt]{beamer}
\usepackage[T1]{fontenc}
\usepackage[utf8]{inputenc}
\usepackage{lmodern}

\usepackage{amsfonts,amsmath}
\allowdisplaybreaks


%% X color
\definecolor{darkblueX}{RGB}{0,62,92}
\definecolor{lightblueX}{RGB}{0,104,128}
\definecolor{verylightblueX}{RGB}{212,232,239}
\definecolor{redX}{RGB}{169,32,33}

%% Set new default colors
\setbeamercolor{title}{fg=white, bg=darkblueX}
\setbeamercolor{frametitle}{fg=darkblueX, bg=white}
\setbeamercolor{block body}{bg=verylightblueX}%
\setbeamercolor{block title}{bg=darkblueX,fg=white}%
\setbeamercolor{structure}{fg=darkblueX,bg=white}
\setbeamercolor{normal text}{fg=darkblueX,bg=white}%

\setbeamercolor{alerted text}{fg=redX}


%% Font

\setbeamerfont{framesubtitle}{size=\tiny}
\setbeamerfont{alerted text}{shape=\bfseries}




%\usepackage[export]{adjustbox}

% Tighter itemize
\expandafter\def\expandafter\normalsize\expandafter{%
    \normalsize%
    \setlength\abovedisplayskip{1pt}%
    \setlength\belowdisplayskip{1pt}%
    \setlength\abovedisplayshortskip{1pt}%
    \setlength\belowdisplayshortskip{1pt}%
}

\expandafter\def\expandafter\small\expandafter{%
    \small%
    \setlength\abovedisplayskip{1pt}%
    \setlength\belowdisplayskip{1pt}%
    \setlength\abovedisplayshortskip{1pt}%
    \setlength\belowdisplayshortskip{1pt}%
}

%\usepackage{beamerX}
\newcommand{\alertmath}[1]{\textcolor{redX}{\boldsymbol{#1}}}
%\setboolean{displaylogo}{false}

\AtBeginSection[]
{
\begin{frame}
        \frametitle{Outline}
        \tableofcontents[currentsection]
   \end{frame}
}
\AtBeginSubsection[]
{
\begin{frame}
        \frametitle{Outline}
        \tableofcontents[currentsection,currentsubsection]
   \end{frame}
}

\setbeamertemplate{frametitle continuation}[from second][]

\usepackage{graphicx}%
\graphicspath{{./}{./Figures/}}
\newcommand{\parinc}[2]{\parbox[c]{#1}{\includegraphics[width=#1]{#2}}}
\newcommand{\parinch}[3]{\parbox[c][#2]{#1}{\includegraphics[width=#1]{#3}}}
\newcommand{\parincb}[3]{\parbox[c]{#1}{\includegraphics[width=#1,height=#2]{#3}}}

\usepackage{color}
\newcommand{\red}[1]{\textcolor{red}{#1}}

\usepackage{tikz}


\usepackage{scalerel,stackengine}
\stackMath
\newcommand\reallywidehat[1]{%
\savestack{\tmpbox}{\stretchto{%
  \scaleto{%
    \scalerel*[\widthof{\ensuremath{#1}}]{\kern-.6pt\bigwedge\kern-.6pt}%
    {\rule[-\textheight/2]{1ex}{\textheight}}%WIDTH-LIMITED BIG WEDGE
  }{\textheight}% 
}{0.5ex}}%
\stackon[1pt]{#1}{\tmpbox}%
}

\newcommand{\grad}{\nabla}
\newcommand{\ind}[1]{\mathbf{1}_{#1}}

\DeclareMathOperator{\limSup}{limsup}%
\DeclareMathOperator{\limInf}{liminf}%
\DeclareMathOperator*{\argmax}{argmax}%
\DeclareMathOperator*{\argmin}{argmin}%
\DeclareMathOperator*{\supp}{Supp}%
\DeclareMathOperator{\pen}{pen}
\DeclareMathOperator{\dom}{dom}
\DeclareMathOperator{\price}{price}
\DeclareMathOperator{\sign}{sign}
\DeclareMathOperator{\KL}{KL}
\DeclareMathOperator{\Proj}{Proj}
\DeclareMathOperator{\Span}{span}


\DeclareMathOperator*{\tr}{tr}
\DeclareMathOperator*{\ra}{rank}
\DeclareMathOperator*{\conv}{conv}
\DeclareMathOperator{\ve}{vec}
\DeclareMathOperator{\diag}{diag}


\newcommand{\Proba}{\mathbb{P}}
\newcommand{\Espe}{\mathbb{E}}
\newcommand{\Vari}{\mathbb{V}}
\newcommand{\Cova}{\mathbb{C}\text{ov}}
\newcommand{\Prob}[2][]{\Proba_{#1}\left\{#2\right\}}
\newcommand{\Esp}[2][]{\Espe_{#1}\left[#2\right]}
\newcommand{\Var}[2][]{\Vari_{#1}\left[#2\right]}
\newcommand{\Cov}[2][]{\Cova_{#1}\left[#2\right]}
\newcommand{\ud}{\textup{d}}
\newcommand{\charac}{\mathbf{1}}

\newcommand{\vecX}{\textbf{X}}
\newcommand{\vecx}{\textbf{x}}
\newcommand{\transp}[1]{{#1}^t}
\usepackage{pdfpages}
\usepackage{tikz}
\usetikzlibrary{positioning}
\usetikzlibrary{arrows}
\tikzset{
	treenode/.style = {align=center, inner sep=0pt, text centered,
		font=\sffamily},
	arn_n/.style = {treenode, circle, black, draw=black,
		fill=white, text width=1.5em, thin},% arbre rouge noir, noeud noir
	arn_r/.style = {treenode, circle, red, draw=red, 
		text width=1.5em, very thick},% arbre rouge noir, noeud rouge
	arn_b/.style = {treenode, circle,  draw=green, 
		text width=1.5em, very thick},% arbre rouge noir, noeud rouge
	arn_x/.style = {treenode, rectangle, draw=black, thick,
		minimum width=1.5em, minimum height=1.5em},% arbre rouge noir, nil
	arn_Leaf/.style = {treenode, fill = red, minimum size = 12pt, inner sep = 2pt, rectangle, draw=black, very thick, 
		minimum width=1.5em, minimum height=1.5em},% arbre rouge noir, nil
	emph/.style={edge from parent/.style={blue,thick,draw}},
	norm/.style={edge from parent/.style={black,thin,draw}}
}


\def\layersep{2.5cm}
\def\HLOneN{5}
\def\HLTwoN{6}
\def\nFeats{2}
\title[Machine Learning]{Machine Learning II\\
	\small
	Neural Networks
}

\author{E. Scornet and E. Le Pennec}

\date{Winter 2017}

\begin{document}





\vspace{-2cm}
\hspace{1.5cm}
\begin{tikzpicture}[shorten >=0pt,-,draw=black, node distance=\layersep, transform canvas={scale=0.7}]
\tikzstyle{every pin edge}=[-, draw = black!30]
\tikzstyle{neuron}=[circle,  draw=black,
fill=white, minimum size=14pt,  inner sep=0pt,  font=\sffamily]

\tikzstyle{input neuron}=[neuron];
\tikzstyle{branch neuron}=[neuron, draw=green, very thick, fill = white, minimum size = 16pt, inner sep = 2pt]
\tikzstyle{invisible neuron}=[circle, draw = white]

\tikzstyle{leaf neuron}=[rectangle,  draw=black,
fill=white, minimum size=14pt,  inner sep=0pt,  font=\sffamily]


\tikzstyle{special leaf neuron} = [rectangle, draw=black, very thick, fill = red, minimum size = 16pt, inner sep = 2pt]
\tikzstyle{output neuron}=[circle,  draw=red,
fill=white, minimum size=14pt,  inner sep=0pt,  font=\sffamily];
\tikzstyle{output neuron bis}==[rectangle,  draw=black,
fill=white, minimum size=14pt,  inner sep=4pt,  font=\sffamily];

\tikzstyle{hidden neuron}=[neuron];
\tikzstyle{annot} = [text width=8em, text centered, font = \footnotesize]


% Draw the input layer nodes
\node[input neuron] (I-1) at (\layersep,-1.5) {$x_{1}$};
\node[input neuron] (I-2) at (\layersep,-2.5) {$x_{2}$};
\node[invisible neuron] (I-3) at (\layersep,-3.5)  {$\hdots$};
\node[input neuron] (I-d) at (\layersep,-4.5) {$x_{d}$};

%Define the bias node
\node[leaf neuron] (I-b) at (1.8*\layersep,-0.7) {$b$};


% Draw the hidden layer nodes HL1

\path[yshift=0.5cm]
node[hidden neuron] (HL1-3) at (2.5*\layersep,-3.5 cm) {};
\node[invisible neuron] (II-1) at (3.2*\layersep,-3cm) {};
% Connect every node in the input layer with every node in the
% hidden layer.


\draw (I-1) -- (HL1-3) node [midway, above] (TextNode) {$w_1$};
\draw (I-2) -- (HL1-3) node [midway, above] (TextNode) {$w_2$};
\draw (I-d) -- (HL1-3) node [midway, above] (TextNode) {$w_d$};
\draw (I-b) -- (HL1-3) node [midway, right] (TextNode) {$b$};	
\draw [->] (HL1-3) -- (II-1) node [above] (TextNode) {$\sigma(\sum_{i=1}^d w_i x_i +b )$};	
% Annotate the layers
\node[annot,above of=HL1-3, node distance=3.2cm] (hl1) { Output layer};
\node[annot,left of=hl1, node distance=3.9cm] {Input layer};


\end{tikzpicture}
\end{document}